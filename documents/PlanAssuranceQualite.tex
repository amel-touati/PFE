\documentclass[11pt,fleqn]{book} % Default font size and left-justified equations

\usepackage[top=3cm,bottom=3cm,left=3.2cm,right=3.2cm,headsep=10pt,letterpaper]{geometry} % Page margins

\usepackage{xcolor} % Required for specifying colors by name
\definecolor{ocre}{RGB}{52,177,201} % Define the orange color used for highlighting throughout the book
\usepackage{wallpaper}
\usepackage{mdframed}
\usepackage[top=2cm, bottom=2cm, outer=0cm, inner=0cm]{geometry}
% Font Settings
\usepackage{avant} % Use the Avantgarde font for headings
%\usepackage{times} % Use the Times font for headings
\usepackage{mathptmx} % Use the Adobe Times Roman as the default text font together with math symbols from the Sym­bol, Chancery and Computer Modern fonts
\graphicspath{ {figures/} }
\usepackage{array}
\usepackage[T1]{fontenc}
\usepackage{imakeidx}
\makeindex
\usepackage[totoc]{idxlayout}
\usepackage{tabularx}
\usepackage{caption}
\usepackage{microtype} % Slightly tweak font spacing for aesthetics
\usepackage[utf8]{inputenc} % Required for including letters with accents
\usepackage[T1]{fontenc} % Use 8-bit encoding that has 256 glyphs
\usepackage{hyperref}
% Bibliography
\usepackage[style=alphabetic,sorting=nyt,sortcites=true,autopunct=true,babel=hyphen,hyperref=true,abbreviate=false,backref=true,backend=biber]{biblatex}
\addbibresource{bibliography.bib} % BibTeX bibliography file
\defbibheading{bibempty}{}

%----------------------------------------------------------------------------------------
%	VARIOUS REQUIRED PACKAGES
%----------------------------------------------------------------------------------------

\usepackage{titlesec} % Allows customization of titles

\usepackage{graphicx} % Required for including pictures
\graphicspath{{Pictures/}} % Specifies the directory where pictures are stored

\usepackage{lipsum} % Inserts dummy text

\usepackage{tikz} % Required for drawing custom shapes

\usepackage[english]{babel} % English language/hyphenation

\usepackage{enumitem} % Customize lists
\setlist{nolistsep} % Reduce spacing between bullet points and numbered lists

\usepackage{booktabs} % Required for nicer horizontal rules in tables

\usepackage{eso-pic} % Required for specifying an image background in the title page

%----------------------------------------------------------------------------------------
%	MAIN TABLE OF CONTENTS
%----------------------------------------------------------------------------------------

\usepackage{titletoc} % Required for manipulating the table of contents

\contentsmargin{0cm} % Removes the default margin
% Chapter text styling
\titlecontents{chapter}[1.25cm] % Indentation
{\addvspace{15pt}\large\sffamily\bfseries} % Spacing and font options for chapters
{\color{ocre!60}\contentslabel[\Large\thecontentslabel]{1.25cm}\color{ocre}} % Chapter number
{}  
{\color{ocre!60}\normalsize\sffamily\bfseries\;\titlerule*[.5pc]{.}\;\thecontentspage} % Page number
% Section text styling
\titlecontents{section}[1.25cm] % Indentation
{\addvspace{5pt}\sffamily\bfseries} % Spacing and font options for sections
{\contentslabel[\thecontentslabel]{1.25cm}} % Section number
{}
{\sffamily\hfill\color{black}\thecontentspage} % Page number
[]
% Subsection text styling
\titlecontents{subsection}[1.25cm] % Indentation
{\addvspace{1pt}\sffamily\small} % Spacing and font options for subsections
{\contentslabel[\thecontentslabel]{1.25cm}} % Subsection number
{}
{\sffamily\;\titlerule*[.5pc]{.}\;\thecontentspage} % Page number
[] 

%----------------------------------------------------------------------------------------
%	MINI TABLE OF CONTENTS IN CHAPTER HEADS
%----------------------------------------------------------------------------------------

% Section text styling
\titlecontents{lsection}[0em] % Indendating
{\footnotesize\sffamily} % Font settings
{}
{}
{}

% Subsection text styling
\titlecontents{lsubsection}[.5em] % Indentation
{\normalfont\footnotesize\sffamily} % Font settings
{}
{}
{}
 
%----------------------------------------------------------------------------------------
%	PAGE HEADERS
%----------------------------------------------------------------------------------------

\usepackage{fancyhdr} % Required for header and footer configuration

\pagestyle{fancy}
\renewcommand{\chaptermark}[1]{\markboth{\sffamily\normalsize\bfseries\chaptername\ \thechapter.\ #1}{}} % Chapter text font settings
\renewcommand{\sectionmark}[1]{\markright{\sffamily\normalsize\thesection\hspace{5pt}#1}{}} % Section text font settings
\fancyhf{} \fancyhead[LE,RO]{\sffamily\normalsize\thepage} % Font setting for the page number in the header
\fancyhead[LO]{\rightmark} % Print the nearest section name on the left side of odd pages
\fancyhead[RE]{\leftmark} % Print the current chapter name on the right side of even pages
\renewcommand{\headrulewidth}{0.5pt} % Width of the rule under the header
\addtolength{\headheight}{2.5pt} % Increase the spacing around the header slightly
\renewcommand{\footrulewidth}{0pt} % Removes the rule in the footer
\fancypagestyle{plain}{\fancyhead{}\renewcommand{\headrulewidth}{0pt}} % Style for when a plain pagestyle is specified

% Removes the header from odd empty pages at the end of chapters
\makeatletter
\renewcommand{\cleardoublepage}{
\clearpage\ifodd\c@page\else
\hbox{}
\vspace*{\fill}
\thispagestyle{empty}
\newpage
\fi}

%----------------------------------------------------------------------------------------
%	THEOREM STYLES
%----------------------------------------------------------------------------------------

\usepackage{amsmath,amsfonts,amssymb,amsthm} % For math equations, theorems, symbols, etc

\newcommand{\intoo}[2]{\mathopen{]}#1\,;#2\mathclose{[}}
\newcommand{\ud}{\mathop{\mathrm{{}d}}\mathopen{}}
\newcommand{\intff}[2]{\mathopen{[}#1\,;#2\mathclose{]}}
\newtheorem{notation}{Notation}[chapter]

%%%%%%%%%%%%%%%%%%%%%%%%%%%%%%%%%%%%%%%%%%%%%%%%%%%%%%%%%%%%%%%%%%%%%%%%%%%
%%%%%%%%%%%%%%%%%%%% dedicated to boxed/framed environements %%%%%%%%%%%%%%
%%%%%%%%%%%%%%%%%%%%%%%%%%%%%%%%%%%%%%%%%%%%%%%%%%%%%%%%%%%%%%%%%%%%%%%%%%%
\newtheoremstyle{ocrenumbox}% % Theorem style name
{0pt}% Space above
{0pt}% Space below
{\normalfont}% % Body font
{}% Indent amount
{\small\bf\sffamily\color{ocre}}% % Theorem head font
{\;}% Punctuation after theorem head
{0.25em}% Space after theorem head
{\small\sffamily\color{ocre}\thmname{#1}\nobreakspace\thmnumber{\@ifnotempty{#1}{}\@upn{#2}}% Theorem text (e.g. Theorem 2.1)
\thmnote{\nobreakspace\the\thm@notefont\sffamily\bfseries\color{black}---\nobreakspace#3.}} % Optional theorem note
\renewcommand{\qedsymbol}{$\blacksquare$}% Optional qed square

\newtheoremstyle{blacknumex}% Theorem style name
{5pt}% Space above
{5pt}% Space below
{\normalfont}% Body font
{} % Indent amount
{\small\bf\sffamily}% Theorem head font
{\;}% Punctuation after theorem head
{0.25em}% Space after theorem head
{\small\sffamily{\tiny\ensuremath{\blacksquare}}\nobreakspace\thmname{#1}\nobreakspace\thmnumber{\@ifnotempty{#1}{}\@upn{#2}}% Theorem text (e.g. Theorem 2.1)
\thmnote{\nobreakspace\the\thm@notefont\sffamily\bfseries---\nobreakspace#3.}}% Optional theorem note

\newtheoremstyle{blacknumbox} % Theorem style name
{0pt}% Space above
{0pt}% Space below
{\normalfont}% Body font
{}% Indent amount
{\small\bf\sffamily}% Theorem head font
{\;}% Punctuation after theorem head
{0.25em}% Space after theorem head
{\small\sffamily\thmname{#1}\nobreakspace\thmnumber{\@ifnotempty{#1}{}\@upn{#2}}% Theorem text (e.g. Theorem 2.1)
\thmnote{\nobreakspace\the\thm@notefont\sffamily\bfseries---\nobreakspace#3.}}% Optional theorem note


\newtheoremstyle{ocrenum}% % Theorem style name
{5pt}% Space above
{5pt}% Space below
{\normalfont}% % Body font
{}% Indent amount
{\small\bf\sffamily\color{ocre}}% % Theorem head font
{\;}% Punctuation after theorem head
{0.25em}% Space after theorem head
{\small\sffamily\color{ocre}\thmname{#1}\nobreakspace\thmnumber{\@ifnotempty{#1}{}\@upn{#2}}% Theorem text (e.g. Theorem 2.1)
\thmnote{\nobreakspace\the\thm@notefont\sffamily\bfseries\color{black}---\nobreakspace#3.}} % Optional theorem note
\renewcommand{\qedsymbol}{$\blacksquare$}% Optional qed square
\makeatother

% Defines the theorem text style for each type of theorem to one of the three styles above
\newcounter{dummy} 
\numberwithin{dummy}{section}
\theoremstyle{ocrenumbox}
\newtheorem{theoremeT}[dummy]{Theorem}
\newtheorem{problem}{Problem}[chapter]
\newtheorem{exerciseT}{Exercise}[chapter]
\theoremstyle{blacknumex}
\newtheorem{exampleT}{Example}[chapter]
\theoremstyle{blacknumbox}
\newtheorem{vocabulary}{Vocabulary}[chapter]
\newtheorem{definitionT}{Definition}[section]
\newtheorem{corollaryT}[dummy]{Corollary}
\theoremstyle{ocrenum}
\newtheorem{proposition}[dummy]{Proposition}

%----------------------------------------------------------------------------------------
%	DEFINITION OF COLORED BOXES
%----------------------------------------------------------------------------------------

\RequirePackage[framemethod=default]{mdframed} % Required for creating the theorem, definition, exercise and corollary boxes

% Theorem box
\newmdenv[skipabove=7pt,
skipbelow=7pt,
backgroundcolor=black!5,
linecolor=ocre,
innerleftmargin=5pt,
innerrightmargin=5pt,
innertopmargin=5pt,
leftmargin=0cm,
rightmargin=0cm,
innerbottommargin=5pt]{tBox}

% Exercise box	  
\newmdenv[skipabove=7pt,
skipbelow=7pt,
rightline=false,
leftline=true,
topline=false,
bottomline=false,
backgroundcolor=ocre!10,
linecolor=ocre,
innerleftmargin=5pt,
innerrightmargin=5pt,
innertopmargin=5pt,
innerbottommargin=5pt,
leftmargin=0cm,
rightmargin=0cm,
linewidth=4pt]{eBox}	

% Definition box
\newmdenv[skipabove=7pt,
skipbelow=7pt,
rightline=false,
leftline=true,
topline=false,
bottomline=false,
linecolor=ocre,
innerleftmargin=5pt,
innerrightmargin=5pt,
innertopmargin=0pt,
leftmargin=0cm,
rightmargin=0cm,
linewidth=4pt,
innerbottommargin=0pt]{dBox}	

% Corollary box
\newmdenv[skipabove=7pt,
skipbelow=7pt,
rightline=false,
leftline=true,
topline=false,
bottomline=false,
linecolor=gray,
backgroundcolor=black!5,
innerleftmargin=5pt,
innerrightmargin=5pt,
innertopmargin=5pt,
leftmargin=0cm,
rightmargin=0cm,
linewidth=4pt,
innerbottommargin=5pt]{cBox}


\newenvironment{theorem}{\begin{tBox}\begin{theoremeT}}{\end{theoremeT}\end{tBox}}
\newenvironment{exercise}{\begin{eBox}\begin{exerciseT}}{\hfill{\color{ocre}\tiny\ensuremath{\blacksquare}}\end{exerciseT}\end{eBox}}				  
\newenvironment{definition}{\begin{dBox}\begin{definitionT}}{\end{definitionT}\end{dBox}}	
\newenvironment{example}{\begin{exampleT}}{\hfill{\tiny\ensuremath{\blacksquare}}\end{exampleT}}		
\newenvironment{corollary}{\begin{cBox}\begin{corollaryT}}{\end{corollaryT}\end{cBox}}	

%----------------------------------------------------------------------------------------
%	REMARK ENVIRONMENT
%----------------------------------------------------------------------------------------

\newenvironment{remark}{\par\vspace{10pt}\small % Vertical white space above the remark and smaller font size
\begin{list}{}{
\leftmargin=35pt % Indentation on the left
\rightmargin=25pt}\item\ignorespaces % Indentation on the right
\makebox[-2.5pt]{\begin{tikzpicture}[overlay]
\node[draw=ocre!60,line width=1pt,circle,fill=ocre!25,font=\sffamily\bfseries,inner sep=2pt,outer sep=0pt] at (-15pt,0pt){\textcolor{ocre}{R}};\end{tikzpicture}} % Orange R in a circle
\advance\baselineskip -1pt}{\end{list}\vskip5pt} % Tighter line spacing and white space after remark

%----------------------------------------------------------------------------------------
%	SECTION NUMBERING IN THE MARGIN
%----------------------------------------------------------------------------------------

\makeatletter
\renewcommand{\@seccntformat}[1]{\llap{\textcolor{ocre}{\csname the#1\endcsname}\hspace{1em}}}                    
\renewcommand{\section}{\@startsection{section}{1}{\z@}
{-4ex \@plus -1ex \@minus -.4ex}
{1ex \@plus.2ex }
{\normalfont\large\sffamily\bfseries}}
\renewcommand{\subsection}{\@startsection {subsection}{2}{\z@}
{-3ex \@plus -0.1ex \@minus -.4ex}
{0.5ex \@plus.2ex }
{\normalfont\sffamily\bfseries}}
\renewcommand{\subsubsection}{\@startsection {subsubsection}{3}{\z@}
{-2ex \@plus -0.1ex \@minus -.2ex}
{.2ex \@plus.2ex }
{\normalfont\small\sffamily\bfseries}}                        
\renewcommand\paragraph{\@startsection{paragraph}{4}{\z@}
{-2ex \@plus-.2ex \@minus .2ex}
{.1ex}
{\normalfont\small\sffamily\bfseries}}

%----------------------------------------------------------------------------------------
%	HYPERLINKS IN THE DOCUMENTS
%----------------------------------------------------------------------------------------

% For an unclear reason, the package should be loaded now and not later
\usepackage{hyperref}
\hypersetup{hidelinks,backref=true,pagebackref=true,hyperindex=true,colorlinks=false,breaklinks=true,urlcolor= ocre,bookmarks=true,bookmarksopen=false,pdftitle={Title},pdfauthor={Author}}

%----------------------------------------------------------------------------------------
%	CHAPTER HEADINGS
%----------------------------------------------------------------------------------------



\newcommand{\thechapterimage}{}
\newcommand{\chapterimage}[1]{\renewcommand{\thechapterimage}{#1}}

% Numbered chapters with mini tableofcontents
\def\thechapter{\arabic{chapter}}
\def\@makechapterhead#1{
\thispagestyle{empty}
{\centering \normalfont\sffamily
\ifnum \c@secnumdepth >\m@ne
\if@mainmatter
\startcontents
\begin{tikzpicture}[remember picture,overlay]
\node at (current page.north west)
{\begin{tikzpicture}[remember picture,overlay]
\node[anchor=north west,inner sep=0pt] at (0,0) {\includegraphics[width=\paperwidth]{\thechapterimage}};
%%%%%%%%%%%%%%%%%%%%%%%%%%%%%%%%%%%%%%%%%%%%%%%%%%%%%%%%%%%%%%%%%%%%%%%%%%%%%%%%%%%%%
% Commenting the 3 lines below removes the small contents box in the chapter heading
%\fill[color=ocre!10!white,opacity=.6] (1cm,0) rectangle (8cm,-7cm);
%\node[anchor=north west] at (1.1cm,.35cm) {\parbox[t][8cm][t]{6.5cm}{\huge\bfseries\flushleft \printcontents{l}{1}{\setcounter{tocdepth}{2}}}};
\draw[anchor=west] (5cm,-9cm) node [rounded corners=20pt,fill=ocre!10!white,text opacity=1,draw=ocre,draw opacity=1,line width=1.5pt,fill opacity=.6,inner sep=12pt]{\huge\sffamily\bfseries\textcolor{black}{\thechapter. #1\strut\makebox[22cm]{}}};
%%%%%%%%%%%%%%%%%%%%%%%%%%%%%%%%%%%%%%%%%%%%%%%%%%%%%%%%%%%%%%%%%%%%%%%%%%%%%%%%%%%%%
\end{tikzpicture}};
\end{tikzpicture}}
\par\vspace*{230\p@}
\fi
\fi}

% Unnumbered chapters without mini tableofcontents (could be added though) 
\def\@makeschapterhead#1{
\thispagestyle{empty}
{\centering \normalfont\sffamily
\ifnum \c@secnumdepth >\m@ne
\if@mainmatter
\begin{tikzpicture}[remember picture,overlay]
\node at (current page.north west)
{\begin{tikzpicture}[remember picture,overlay]
\node[anchor=north west,inner sep=0pt] at (0,0) {\includegraphics[width=\paperwidth]{\thechapterimage}};
\draw[anchor=west] (5cm,-9cm) node [rounded corners=20pt,fill=ocre!10!white,fill opacity=.6,inner sep=12pt,text opacity=1,draw=ocre,draw opacity=1,line width=1.5pt]{\huge\sffamily\bfseries\textcolor{black}{#1\strut\makebox[22cm]{}}};
\end{tikzpicture}};
\end{tikzpicture}}
\par\vspace*{230\p@}
\fi
\fi
}
\makeatother
 % Insert the commands.tex file which contains the majority of the structure behind the template

\begin{document}
\title{Backlogs}

%----------------------------------------------------------
%	TITLE PAGE
%----------------------------------------------------------
\begingroup
\ThisLRCornerWallPaper{1.0}{Pictures/cover page.png}
\endgroup

%----------------------------------------Version page ---------

\newpage
~\vfill
\thispagestyle{empty}

%-------------------	TABLE OF CONTENTS

\pagestyle{empty} % No headers

\tableofcontents

\pagestyle{fancy} % Print headers again

%------------------------------------------------------------------------%
\chapter{Environnement}
\section{Introduction}
Le Plan d’Assurance Qualité de Projet est un document décrivant comment mettre en œuvre les moyens permettant d’obtenir la qualité nécessaire à la bonne réalisation d’un projet. Il précise également les dispositions relatives à la conception et à la maitrise de la qualité pour l’ensemble du cycle de vie d’un système. 
\section{Présentation et But du Projet}
L’Ecole Supérieure en Informatique 08 mai 1945 de Sidi Bel Abbes est un établissement universitaire public, placé sous la tutelle du Ministère de l'Enseignement Supérieur et de la Recherche Scientifique, créé par le décret exécutif n° 14-232 du 25 août 2014. \\
L'école décerne le double diplôme d'ingénieur d’état et de master en informatique après une formation de cinq ans. Deux années préparatoires, une année de tronc commun et deux années de spécialisation. \\
Deux spécialités sont actuellement habilitées : 
\begin{itemize}
    \item Systèmes d'information et Web (SIW).
    \item Ingénierie des systèmes informatiques (ISI).
\end{itemize}
Notre projet consiste à gérer les projets de fin d’études effectué au niveau de cette école.
\newpage
\section{Equipe et Clients}
\newpage
\section{Rôles des Membres d’Équipe}
\newpage
\section{Objectifs du Document}
\begin{itemize}
    \item La structuration globale du projet et son plan de développement détaillé.
    \item Définition du plan de gestion du projet.
    \item La distribution des responsabilités entre les membres dans la structure, les plans de développement et de gestion du projet. 
    \tem Définition des barrières imposées au projet.
\end{itemize}
\section{ Procédure d’Évaluation du PAQ}
Créée au début du projet, le PAQL est un document évolutif, un outil de suivi et de gestion, il peut subir diverses modifications car il est utilisé tout au long du projet avec intégration des nouvelles données donc Il doit être flexible pour permettre un ajustement constant dépendant des circonstances ajustées en fonction des circonstances.\\
Chaque modification sera indiquée dans la table de l’historique.
\chapter{Pilotage du projet}
\section{les phases du projet}
\subsection{ Phase d’initialisation}
\subsubsection{Durée}
De 06-03-2022 jusqu’à 13-03-2022
\subsubsection{Objectifs}
\begin{itemize}
    \item  Déterminer les choix techniques pour résoudre la problématique.
    \item  Proposer une maquette globale du système.
    \item  Préciser l’objectif de projet, et identifier les acteurs.
\end{itemize}
\subsubsection{Difficulté rencontrés}
\begin{itemize}
    \item  S’habituer aux membres de l’équipe.
\end{itemize}
\subsubsection{Livrables et responsabilités }
\begin{itemize}
    \item Profils de l’équipe
    \item Brochure
    \item Charte de codage
    \item Charte de nommage
    \item Charte de document
    \item Plan Assurance Qualité
    \item Workflow de validation
    \item Planning Prévisionnel
\end{itemize}

\subsection{Phase de développement}
\subsubsection{Durée}
De 15-03-2022 jusqu’à 30-05-2022
\subsubsection{Objectifs}
\begin{itemize}
    \item Définir les besoins dans un cahier de charges.
    \item Etude analytique et conceptuelle du système.
    \item Définir l’architecture du système
    \item Développer un prototype on lui effectuant des tests
    \item  Définir les plans de test. 
\end{itemize}
\subsubsection{Difficulté rencontrés}
\begin{itemize}
    \item  Dépendance entre les livrables. 
\end{itemize}
\subsubsection{Livrables et responsabilités}
\begin{itemize}
    \item Cahier de charges
    \item Rapport d’architecture
    \item Rapport d’analyse
    \item Rapport de conception
    \item Plan de test initial
    \item Code Source
    \item Manuels
    \item Plan de tests final
    \item Rapports de tests
\end{itemize}


\subsection{Phase de bilan}
\subsubsection{Durée}
De 9-06-2022 jusqu’à 14-06-2022
\subsubsection{Objectifs}
\begin{itemize}
    \item  Evaluer le déroulement du projet.
\end{itemize}
\subsubsection{Difficultés rencontrées}
\begin{itemize}
    \item 
\end{itemize}
\subsubsection{Livrables et responsabilités}
\begin{itemize}
    \item Compte rendu global sur le projet.
\end{itemize}


\subsection{ Suivi des Tâches}
\subsubsection{Durée}
Périodiquement (chaque semaine ou deux semaines).
 \subsubsection{Objectifs}
\begin{itemize}
    \item  Voir l’état d’avancement de projet
\end{itemize}
\subsubsection{Difficultés rencontrées}
\begin{itemize}
    \item 
\end{itemize}
\subsubsection{Livrables et responsabilités}
\begin{itemize}
    \item Comptes rendus des réunions 
    \item  Suivi des tâches
    \item Suivi du moral 
\end{itemize}

\section{Organisation des Réunions}
Le contrôle global du projet est à la charge des différentes structures du projet. Il est organisé dans le cadre des réunions périodiques entre les membres de l’équipe “ ItExperts“.\\ 
Chaque réunion de structure fait l’objet d’un ordre du jour (qui peut être implicite) et d’un compte-rendu qui contient les informations nécessaire de la réunion (présence, les tache a faire pour la prochain réunion,…).
\section{Procédure de Validation des Documents de 
Projet}
Le cycle de vie de validation de nos documents de projet est de passer en premier au 
chef de qualité qui les vérifier puis les envoyer au chef de projet qui lui aussi vérifier 
ces documents et les valider. \\
En cas de petit remarques les chefs essayent de les corrigées, sinon ils les renvoient au 
rédacteur pour les corrigées dans un délai de un jour \\
La validation du document est discutée en comité de pilotage, en cas de : 
\begin{itemize}
    \item  dépassement de délais par le rédacteur
    \item  refus des remarques
    \item  refus des corrections
\end{itemize}
La validation est effective en cas de : 
\begin{itemize}
    \item accord du client
    \item dépassement des délais par le destinataire
\end{itemize}
\section{ Missions et Responsabilités}
Les missions et les responsabilités du comité de l’équipe projet portent sur les éléments suivants :
\begin{itemize}
    \item Contrôle de l’utilisation des ressources allouées.
    \item Suivi du consommé et comparaison avec la planification, analyse des écarts.
    \item Présentation des problèmes (techniques, organisationnels, de planning), évaluation des glissements temporels, résolution des problèmes ponctuels.
    \item Coordination de l’équipe projet (pour des actions interdépendantes).
    \item Planning détaillé pour la période proche à venir (identification des tâches,définition précise, ordonnancement).
    \item Recensement des informations nécessitant la décision au niveau du pilotage du projet.
\end{itemize}


\chapter{ Qualité et Tests}
\section{Normalisation du Code Source}
Le code source doit obligatoirement respecter l'ensemble des règles qui sont définie dans la charte de codage pour uniformiser les pratiques de développement logiciel, diffuser les bonnes pratiques de développement et éviter les erreurs au sein d’un groupe.
\section{Tests des sources}
L’évaluation de chaque composant de logiciel pour déterminer s'ils satisfont la spécification élaborée lors de la phase d'analyse passe par des tests qui sont définie dans le plan de test.
\begin{itemize}
    \item Les tests se feront à la fin de chaque module. 
    \item Chacun des développeurs devra vérifier d’abord lui-même la validité de ses sources avant de les 
    soumettre à l’équipe.
    \item Les sources qui ne conviendraient pas au plan de test seront retournées aux développeurs par le  responsable des tests afin d’apporter une correction.
    \item Les tests réalisés seront indiqués dans un document de type plan de test.
\end{itemize}

\chapter{Format des documents}
\section{Comptes-rendus de réunions}
Pour toute réunion, un compte rendu écrit et succinct sera rédigé et approuvé par chaque partie. \\
En cas d’absence de validation écrite du compte-rendu par les parties, le document est considéré comme approuvé définitivement après un délai d’une semaine à dater de la remise du document aux parties:
Le contenu du compte-rendu :
\begin{itemize}
    \item  La date et heure de la réunion.
    \item  Les noms des membres absents et présents.
    \item  Les points discutés dans cette réunion.
    \item  Les taches à réaliser pour la prochaine réunion.
    \item  La date approximative de la prochaine réunion.
\end{itemize}
\section{ Droit de Modification du PAQ}
Tout membre appartenant à la structure, peut, après validation des autres membres, modifier le Plan Qualité de Projet. Cette modification sera alors rapportée dans le 
compte-rendu de la réunion et aussi insérée dans la table des versions du document traité. 

\end{document}
